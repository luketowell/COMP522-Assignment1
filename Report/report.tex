\documentclass[a4paper, twoside, 11pt]{article}
\usepackage[margin=1.5cm]{geometry}
\usepackage[]{xcolor}
\usepackage{cite}
\usepackage{graphicx}
\usepackage{listings}
\usepackage{float}
\usepackage{amsmath}

\lstset{frame=tb,
  language=Java,
  aboveskip=3mm,
  belowskip=3mm,
  showstringspaces=false,
  columns=flexible,
  basicstyle={\small\ttfamily},
  numbers=none,
  numberstyle=\tiny\color{gray},
  keywordstyle=\color{blue},
  commentstyle=\color{dkgreen},
  stringstyle=\color{mauve},
  breaklines=true,
  breakatwhitespace=true,
  tabsize=3
}

\definecolor{dkgreen}{rgb}{0,0.6,0}
\definecolor{gray}{rgb}{0.5,0.5,0.5}
\definecolor{mauve}{rgb}{0.58,0,0.82}

% define the title
\author{L. ~Towell, L.M.Towell@liverpool.ac.uk}

%defines the title
\title{Assignment 1\break Brute Force Attack Estimation}

\begin{document}
	%generate the title
	\maketitle

\maketitle
\section{Password List}
Below is the password list that I have decided to use for this exercise.
\begin{center}
	\begin{tabular}{ |c|c| } 
	 \hline
	 N & Password \\
	 \hline
	 1 & abc \\ 
	 2 & P@ssW0rD \\ 
	 3 & Th!\$IsAV3ryL0n9pA\$\$w0rd \\ 
	 \hline
	\end{tabular}
\end{center}

\section{Salt and Iteration Count}
For the purpose of this coursework I have hardcoded the salt used within this program, 
I have also decided to use an iteration count of 1024. The below table shows the time per run and the average time
 taken in milliseconds over 5 iterations to encrypt and decrypt the string "This is an example string" using the 
 defined salt and iteration counts.

 The below timings have been recorded running the program on a Macbook air with an Intel core i5 processor and 16GB of RAM. Timings on other machines are likely to differ.

\begin{center}
	\begin{tabular}{ |c|c|c|c| } 
	 \hline
	 \multicolumn{4}{|c|}{Iteration time in milliseconds (ms)} \\
	 \hline
	 N & abc & P@ssW0rD & Th!\$IsAV3ryL0n9pA\$\$w0rd \\
	 \hline
	 1 & 1 & 1 & 1 \\ 
	 2 & 1 & 1 & 1  \\ 
	 3 & 1 & 1 & 1  \\ 
	 4 & 1 & 1 & 1  \\
	 5 & 1 & 1 & 1  \\
	 \hline
	 \hline
	 Average & 1 & 1 & 1  \\
	 \hline
	\end{tabular}
\end{center}

\section{Brute Force Attack Estimation}
Given that it is specified that the attacker knows the salt, iteration count, encryption type, 
input string and cipher text used to encrypt the input string then the only piece of information 
the attacker would need to find is the password used to encrypt the string. 
In order to work out the password using a Brute force attach the attacker is going to have to 
iterate through each possible character that could be used in each different combination.

If we presume that passwords are made of uppercase, lowercase numbers and special characters and they are
limited to traditional ASCII encoded characters then this gives the attacker a possible 95 characters for each character in the password.
The equation for working out a password via brute force attack is therefore $95^n$ where n is the number of characters within the password.

There for is we take the time for one iteration from above which is 1ms 

\section{How Does Iteration Count Affect Brute Force Timing?}
Some text

\section{Brute Force Attack Estimation Without Known Iteration Count}
Some text

\section{Comparison with Online Services}
Some text

\newpage
\section*{Appendices}
\appendix
\section{Password Encryption \& Decryption Program}
\lstinputlisting{../code/PasswordBasedEncryption.java}
\newpage
\section{Output of Appendix A when ran in terminal}
\lstinputlisting{../code/timings.txt}

\end{document}